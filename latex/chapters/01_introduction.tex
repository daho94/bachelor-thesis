\section{Einleitung}

\subsection{Motivation und Kontext}
Mit der fortschreitenden Entwicklung von Verkehrsmitteln und der dadurch zunehmenden Mobilität
gewinnt die Routenplanung eine immer größere Bedeutung. Routenplanung ist ein faszinierendes und
herausforderndes Gebiet, das eine wichtige Rolle in verschiedenen Bereichen spielt. Egal, ob es
darum geht, den schnellsten Weg von einem Ort zum anderen zu finden, die effizienteste Route für die
Zustellung von Waren zu bestimmen oder den Verkehr in einem komplexen Straßensystem zu simulieren,
durch die Anwendung und Erforschung von Routenplanungsalgorithmen können effiziente und optimale
Wege in komplexen Netzwerken gefunden werden, um Zeit, Ressourcen und Kosten zu sparen.\\

Das Problem, nach der Suche des kürzesten Weges, lässt sich auf eines der fundamentalsten Probleme
aus der Graphentheorie, einem Teilgebiet der Mathematik und theoretischen Informatik %\todo{Add ref}
zurückführen. Mit Graphen lassen sich eine Vielzahl von Problemen aus der echten Welt als
mathematische Struktur, bestehend aus Knoten und Kanten, modellieren. So kann auch ein
Straßennetzwerk durch Knoten, die Kreuzungen repräsentieren und Kanten, die als Straßensegmente
Kreuzungen miteinander verbinden, dargestellt werden. Jeder Kante wird dabei ein Gewicht zugewiesen,
das die mit dem Durchlaufen dieser Kante verbundenen Kosten widerspiegelt. Auf dem Graphen lassen
sich anschließend Algorithmen ausführen, die den kürzesten Weg dist(s,t) zwischen einem Startknoten
s und Zielknoten t bestimmen können, indem die Kosten des Weges minimiert werden. Einer der wohl
bekanntesten Algorithmen, um diese Aufgabe zu lösen wurde von dem niederländischen Informatiker
Edsger W. Dijkstra entwickelt und im Jahr 1959 veröffentlicht \cite{Dijkstra_1959}. Der
Dijkstra-Algorithmus funktioniert zwar gut auf kleinen Graphen und wird auch noch heutzutage häufig
angewendet, skaliert jedoch schlecht mit immer größer werdenden Datenmengen, denn im schlechtesten
Fall muss der gesamte Graph traversiert werden. Da ein Straßennetz aus mehreren Millionen Knoten und
Kanten besteht, ist der Dijkstra-Algorithmus daher in Anwendungen, die in kurzer Zeit viele kürzeste
Wege berechnen müssen, wie \zB Navigationssysteme, die Routen in Echtzeit aktualisieren müssen,
nicht mehr geeignet.

Um dieses Problem zu lösen, wurden im Laufe der Zeit neue Speed-Up Techniken entwickelt, die die
Laufzeit der Suche verbessern. So wurde \ua der A*-Algorithmus ("`A-Stern"') im Jahr 1968 von Peter
Hart, Nils J. Nilsson und Bertram Raphael als eine Erweiterung des Dijkstra-Algorithmus
veröffentlicht \cite{Hart.1968}. Der Algorithmus ist in der Lage durch das Einführen einer
zusätzlichen Heuristik, orientierter Richtung Ziel zu suchen und damit den Suchraum deutlich
einzuschränken. Dadurch wurde die Laufzeit nochmals verbessert, war aber immer noch nicht
ausreichend für sehr große Graphen.

Viele weitere Techniken basieren auf einer starken
Vorverarbeitung des Graphen. So wurde 2008 von Geisberger, Sanders, Schultes, und Delling
vorgestellt \cite{geisberger.workshop}, die als \acp{CH} bezeichnet wird. Sie basiert auf einer
Vorverarbeitung des Graphen, bei der ausgenutzt wird, dass Straßennetzwerke bereits eine natürliche
Hierarchie besitzen. Während der Vorverarbeitung werden dem Graph zusätzliche Informationen
hinzugefügt, die dann zur Laufzeit während der Suche ausgenutzt werden, was zu erheblich schnelleren
Berechnung der Route führt. Da diese Technik als Sprungbrett für viele neue erweiterte
Routenplanungstechniken gilt, soll sie im Rahmen dieser Arbeit genauer untersucht werden. Dazu soll
ein Prototyp erstellt werden, der die Funktionsweise von \acp{CH} durch eine konkrete
Implementierung demonstriert. Als Eingangsdaten werden die frei nutzbaren Geodaten des
OpenStreetMap-Projekts (Open Data) %\todo{Add ref}%
verwendet, um die Ergebnisse an realen Daten zu
analysieren und zu testen.


\subsection{Stand der Forschung}
Die Berechnung von kürzesten Wegen in dynamischen gewichteten Graphen ist in den letzten Jahrzehnten
intensiv untersucht worden und es entstanden viele neue Techniken zur Lösung verschiedener Varianten
des Problems. Das kürzeste-Wege-Problem ist so relevant, dass regelmäßig Wettbewerbe stattfinden,
bei denen die aktuell besten Routenplanungsalgorithmen auf speziellen Eingabedaten ermittelt werden.
So wurde \zB 2006 die neunte DIMACS Implementation Challenge \cite{dimacs9} ausgerichtet, in der die
"`State of the Art"'-Techniken vorgestellt wurden. Eine ausführliche Übersicht über verschiedene
Algorithmen zur Routenplanung in Straßennetzwerken wurde von Delling et al.
\cite{delling.engineering} veröffentlicht, ist aber durch die signifikante Weiterentwicklungen der
letzten Jahre nicht mehr topaktuell. Es sind neue Algorithmen entstanden, die Suchanfragen auf
Straßennetzwerken in der Größenordnung von Kontinenten in wenigen hundert Nanosekunden beantworten
können oder aktuelle Verkehrsinformationen mit in die Suche einfließen lassen
\cite{Bast.20.04.2015}. Durch den aktuellen Stand der Technik können Methoden des maschinellen
Lernens verwendet werden, um je nach Situation den Verkehrsfluss in Echtzeit vorherzusagen und mit in der
Routenplanung zu berücksichtigen \cite{LIEBIG2017258}.


Diese Entwicklungen zeigen, dass die Routenplanung in Straßennetzwerken
ein aktiver Forschungsbereich ist, der sich ständig weiterentwickelt, um den Bedürfnissen der Nutzer
und modernen Anwendungen gerecht zu werden.

\subsection{Beitrag der Arbeit}
Im Rahmen dieser Arbeit wird die Implementierung der \acp{CH}-Technik in der Programmiersprache
\emph{Rust} untersucht. Rust ist eine moderne, systemsprachenorientierte Programmiersprache, die auf
Performance, Sicherheit und Nebenläufigkeit abzielt \cite{rust.book}. Die Wahl von Rust als
Implementierungssprache bietet die Möglichkeit, die Vorteile dieser Sprache in Bezug auf
Geschwindigkeit, Speichersicherheit und Thread-Sicherheit in der Routenplanung zu erforschen.

Das Hauptziel dieser Arbeit ist, die Implementierung von \acp{CH} detailliert zu beschreiben und
anschließend eine umfassende Leistungsanalyse durchzuführen, indem die Ergebnisse mit herkömmlichen
Algorithmen wie dem Dijkstra-Algorithmus und dem A*-Algorithmus verglichen werden. Die Evaluierung
erfolgt anhand verschiedener Metriken wie Laufzeit, Vorverarbeitungszeit und Speicherbedarf.
Durch diesen Vergleich wird ein tieferes Verständnis für die Leistungsfähigkeit von \acp{CH}
gewonnen und die verbundenen Vor- und Nachteile im Vergleich zu herkömmlichen Algorithmen ermittelt.

Die Ergebnisse dieser Arbeit können wichtige Erkenntnisse liefern, die zur
Weiterentwicklung und Optimierung von Routenplanungssystemen beitragen. Darüber
hinaus bietet die Implementierung in Rust einen wertvollen Beitrag zur
wachsenden Gemeinschaft von Rust-Entwicklern und demonstriert die Anwendung der
Sprache in einem relevanten Anwendungsfall.

\subsection{Struktur der Arbeit}
Die folgenden Abschnitte der Arbeit werden die theoretischen Grundlagen von
Routenplanungsalgorithmen, insbesondere des Dijkstra-Algorithmus, des A*-Algorithmus
und der CH-Technik, erläutern. Anschließend wird die Implementierung der CH-Technik in
Rust beschrieben und detailliert analysiert. Abschließend werden die Ergebnisse der
Leistungsanalyse präsentiert und diskutiert, gefolgt von einem Fazit, das die Erkenntnisse
dieser Arbeit zusammenfasst und mögliche Ansätze für zukünftige Forschungen aufzeigt.
% \todo{Shorten}