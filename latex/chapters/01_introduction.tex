\section{Einleitung}

\subsection{Motivation und Kontext}
Mit der fortschreitenden Entwicklung von Verkehrsmitteln und der dadurch zunehmenden Mobilität
gewinnt die Routenplanung eine immer größere Bedeutung. Routenplanung ist ein faszinierendes und
herausforderndes Gebiet, das eine wichtige Rolle in verschiedenen Bereichen spielt. Egal, ob es
darum geht, den schnellsten Weg von einem Ort zum anderen zu finden, die effizienteste Route für die
Zustellung von Waren zu bestimmen oder den Verkehr in einem komplexen Straßensystem zu simulieren,
durch die Anwendung und Erforschung von Routenplanungsalgorithmen können effiziente und optimale
Wege in komplexen Netzwerken gefunden werden, um Zeit, Ressourcen und Kosten zu sparen.


Das Problem, nach der Suche des kürzesten Weges, lässt sich auf ein zentrales und bekanntes Problem
aus der Graphentheorie, einem Teilgebiet der Mathematik und theoretischen Informatik \todo{Add ref}
zurückführen. Ein Straßennetzwerk kann als Graph, eine mathematische Struktur, die aus Knoten und
Kanten besteht, modelliert werden. Knoten repräsentieren dabei Kreuzungen und Kanten Straßen, die zwei
Knoten miteinander verbinden. Auf dem Graphen lassen sich anschließend Algorithmen ausführen, die
den kürzesten Weg zwischen zwei Knoten bestimmen können. Einer der wohl bekanntesten Algorithmen, um
diese Aufgabe zu lösen wurde von dem niederländischen Informatiker Edsger W. Dijkstra entwickelt und
im Jahr 1959 veröffentlicht \cite{Dijkstra_1959}. Der Dijkstra-Algorithmus funktioniert zwar gut auf
kleinen Graphen und wird auch noch heutzutage häufig angewendet, skaliert jedoch schlecht mit immer
größer werdenden Datenmengen. Da ein Straßennetz aus mehreren Millionen Knoten und Kanten besteht,
ist der Dijkstra-Algorithmus daher in Anwendungen, die in kurzer Zeit viele kürzeste Wege berechnen
müssen, wie \zB Navigationssysteme, die Routen in Echtzeit aktualisieren müssen, nicht mehr geeignet.


Um dieses Problem zu lösen, wurden im Laufe der Zeit neue Speed-Up Techniken entwickelt, die die
Laufzeit der Suche verbessern. So wurde \ua der A*-Algorithmus ("`A-Stern"') im Jahr 1968 von Peter
Hart, Nils J. Nilsson und Bertram Raphael als eine Erweiterung des Dijkstra-Algorithmus
veröffentlicht\todo{Add ref}. Der Algorithmus verfolgt das Ziel durch das Hinzufügen einer
zusätzlichen Heuristik, orientierter Richtung Ziel zu suchen. Dadurch wurde die Laufzeit
nochmals verbessert, war aber immer noch nicht schnell genug für sehr große Graphen.
Eine weitere Technik wurde 2008 von Geisberger, Sanders, Schultes, und Delling vorgestellt
\cite{geisberger.workshop}, die als Contraction Hierarchies (CHs) bezeichnet wird. Sie basiert auf
einer Vorverarbeitung des Graphen, bei der ausgenutzt wird, dass Straßennetzwerke bereits eine
natürliche Hierarchie besitzen. Während der Vorverarbeitung werden dem Graph zusätzliche
Informationen hinzugefügt, die dann zur Laufzeit während der Suche ausgenutzt werden, was zu
erheblich schnelleren Laufzeiten führt. Da diese Technik als Sprungbrett für viele neue erweiterte
Routenplanungstechniken gilt, soll sie im Rahmen dieser Arbeit genauer untersucht werden. Dazu soll
ein Prototyp implementiert werden, der die Funktionsweise von CHs durch eine konkrete
Implementierung demonstriert. Als Eingangsdaten werden die frei nutzbaren Geodaten des
OpenStreetMap-Projekts (Open Data) \todo{Add ref} verwendet, um die Ergebnisse an realen Daten zu
analysieren und testen.


\subsection{Stand der Forschung}
\begin{enumerate}
    \item Wettbewerbe
    \item ...
\end{enumerate}

\subsection{Beitrag der Arbeit}

\subsection{Struktur der Arbeit}