\section{Theoretische und technische Grundlagen}

\subsection{Datenstrukturen}
\subsubsection{Graph}
Um das kürzeste-Wege-Problem zu lösen, muss zunächst das reale Straßennetz in eine abstrakte Form
gebracht werden. Hierzu wird das Straßennetz als Graph modelliert. Ein Graph $G = (V,E)$ besteht aus
einer Menge von Knoten V und einer Menge von Kanten E. Jede Kante $e = (u,v)\in E$ verbindet zwei
Knoten $u,v \in V$. Knoten bilden Kreuzungen ab und Kanten Straßensegmente zwischen zwei Kreuzungen.
Ein Graph kann als gerichtet oder ungerichtet definiert werden. Bei einem ungerichteten  Graphen
sind die Kanten bidirektional und können in beide Richtungen durchlaufen werden, während bei einem
gerichteten Graphen die Kanten nur in eine Richtung durchlaufen werden können.\\

Der Graph in dieser Arbeit ist gerichtet und gewichtet. Die Kantenrichtung spiegelt dabei die
Richtung des Verkehrs wieder, \dH sie gibt an, ob eine Straße in einer bestimmten Richtung befahren
werden darf oder nicht. Zusätzlich wird für jede Kante ein Gewicht $w(e)$ festgelegt. In diesem Fall
wird die durchscnittliche Zeit $t = \frac{V}{S}$ in Sekunden verwendet, die  sich aus der Länge des
Straßensegments \emph{S} und der maximal erlaubten Geschwindigkeit \emph{V} auf dieser Straße
ergibt. Der kürzeste Weg zwischen zwei Knoten ist damit der Weg mit der minimalen Zeit.\\

Es gibt generell zwei Möglichkeiten, einen Graphen als Datenstruktur darzustellen. Eine davon ist
die Verwendung von Adjazenzlisten. Dabei wird für jeden Knoten $u \in V$ eine Liste der benachbarten
Knoten bzw. ausgehenden Kanten $e(u,v) \in E$ gespeichert. Die Summe der Länge aller Adjazenzlisten
in einem gerichteten Graph entspricht der Anzahl an Kanten $|E|$, bzw. $2|E|$ in einem ungerichteten
Graph. Der Speicherverbrauch für die Adjazenzliste ist damit $\Theta(|V| + |E|)$ \cite{intro.algo}.

Die zweite Möglichkeit ist die Verwendung einer Adjazenzmatrix. Hier wird eine zweidimensionale
Matrix der Größe $|V| \times |V|$ verwendet, bei der die Zeilen und Spalten den Knoten entsprechen.
Der Eintrag an Position $(i,j)$ in der Matrix gibt an, ob eine Kante zwischen den Knoten $i$ und $j$
existiert. Unabhängig von der Anzahl an Kanten ist der Speicherverbrauch $\Theta(|V|^2)$
\cite{intro.algo}. Ein Beispiel für beide Darstellungen anhand eines einfachen Graph befindet sich
in Abbildung~\ref{fig:graph_ex1}.\\

Für diese Arbeit werden Adjazenzlisten verwendet, da zum einen Straßennetze dünne Graphen sind ($|E|
	<< |V|^2$), und damit die Speichereffizienz gegenüber einer Adjazenzmatrix deutlich effizienter
ist. Zum anderen ist für es wichtig, für den Aufbau der \ac{CHs} und der Suche, schnell auf alle
Nachbarn eines Knotens zuzugreifen zu können. Der Zugriff auf
eingehende Kanten ist allerding nur sehr aufwendig  möglich, wird  aber für den Aufbau der
Kontraktionshierarchien benötigt. Das Problem lässt sich jedoch lösen indem auch
Adjazenzlisten für alle eingehenden Kanten eines Knotens angelegt werden. Damit erhöht sich der
Speicherverbrauch auf $\Theta(|V| + 2|E|)$, was aber immer noch deutlich effizienter ist als
eine Darstellung als Adjazenzmatrix.\todo{Ref genaue Datenstruktur?}

\begin{figure}[H]
	\centering
	\begin{subfigure}{1.0\textwidth}
		\centering
		\begin{tikzpicture}
			\Vertices[color=gray!90!blue]{data/graphs/ex1/vertices.csv}
			\Edges{data/graphs/ex1/edges.csv}
		\end{tikzpicture}
		\caption{}
	\end{subfigure}
	\\[3ex]
	\begin{subfigure}[b]{0.3\textwidth}
		\centering
		\begin{blockarray}{c|ccccc}
			& A & B & C & D & E \\
			\BAhline
			\begin{block}{c|ccccc}
				A &0 & 3 & 5 & 0 & 3 \\
				B &3 & 0 & 3 & 0 & 0 \\
				C &0 & 0 & 0 & 2 & 0 \\
				D &0 & 0 & 2 & 0 & 3 \\
				E &3 & 0 & 0 & 3 & 0 \\
			\end{block}
		\end{blockarray} \
		\caption{}
	\end{subfigure}
	\hspace{3em}
	\begin{subfigure}[b]{0.3\textwidth}
		\centering
		\begin{tikzpicture}[
				node distance = 21mm and 7mm,
				box/.style = {draw, minimum size=5mm, inner sep=0pt, outer sep=0pt, anchor=west},
				pin edge = {Straight Barb-, shorten <=1mm,semithick}
			]
			\matrix (mat1) [matrix of nodes,
				nodes={box},
				align=left,
				column sep=-\pgflinewidth,
				inner sep=0pt,
				pin=180:A
			]
			{
				B & C & E \\
			};
			\matrix (mat2) [matrix of nodes,
				below=1.5em of mat1-1-1.west,
				anchor=west,
				nodes={box},
				column sep=-\pgflinewidth,
				inner sep=0pt,
				pin=180:B
			]
			{
				A & C \\
			};
			\matrix (mat3) [matrix of nodes,
				below=1.5em of mat2-1-1.west,
				anchor=west,
				nodes={box},
				column sep=-\pgflinewidth,
				inner sep=0pt,
				pin=180:C
			]
			{
				D \\
			};
			\matrix (mat4) [matrix of nodes,
				below=1.5em of mat3-1-1.west,
				anchor=west,
				nodes={box},
				column sep=-\pgflinewidth,
				inner sep=0pt,
				pin=180:D
			]
			{
				C & E \\
			};
			\matrix (mat5) [matrix of nodes,
				below=1.5em of mat4-1-1.west,
				anchor=west,
				nodes={box},
				column sep=-\pgflinewidth,
				inner sep=0pt,
				pin=180:E
			]
			{
				A & D \\
			};
		\end{tikzpicture}
		\vspace{1em}
		\caption{}
	\end{subfigure}
	\caption{Der oben gezeigte gerichtete Graph (a) dargestellt als Adjazenzmatrix(b) oder
		Adjazenzliste (c). Viele Einträge der Matrix sind 0, da der Graph dünn ist, \dH $|E|$
		um einiges kleiner ist als $|V^2|$.}
	\label{fig:graph_ex1}
\end{figure}

\subsubsection{Vorrangwarteschlangen}
Eine \ac{PQ} ist eine Datenstruktur, in der nur auf das Element mit der höchsten Priorität
zugegriffen werden kann. Eine \ac{PQ} unterstützt in der Regel die folgenden Operationen:
\begin{enumerate}
	\item Einfügen (Push): Ein Element wird mit seiner zugehörigen Priorität in die Warteschlange
	      eingefügt. Das Element wird entsprechend seiner Priorität platziert.
	\item Entfernen (Pop): Das Element mit der aktuell höchsten Priorität wird aus der Warteschlange
	      entfernt und zurückgegeben.
\end{enumerate}
\ac{PQ}s werden allen nachfolgenden Suchalgorithmen verwendet, daher hat die Implementierung der
Warteschlange einen großen Einfluss auf die Laufzeit der Algorithmen. In der Arbeit wird die
Implementierung als binärern Min-Heap verwendet mit einer Zeitkompexität für das Einfügen von
$\theta(1)\sim$ und für das Entfernen von $\theta(\log n)$.

\subsection{Kürzeste-Wege-Algorithmen}
Grundsätzlich arbeiten kürzeste-Wege-Algorithmen daran, den kürzesten Weg zwischen einem Startknoten
$s$ und einem Zielknoten $t$ in einem gewichteten Graph zu finden. Der kürzeste Weg $P$ bezieht sich
dabei auf den Weg mit dem geringsten Gesamtgewicht $dist(s,t)$, der sich aus der Summe der Kosten
zum Überqueren der einzelnen Kanten ergibt. Neben dem Punkt-zu-Punkt-kürzeste-Wege-Problem
existieren noch weitere Varianten, wie das Eins-zu-Viele-Problem, bei dem der kürzeste Weg von einem
Knoten $s$ zu allen anderen Knoten im Graphen gesucht wird, oder das Viele-zu-Viele-Problem, bei dem
jeder kürzeste Weg zwischen einer Knotenmenge $S$ und einer Knotenmenge $T$ gesucht wird
\cite{Bast.20.04.2015}. In dieser Arbeit wird sich hauptsächlich auf das Punkt-zu-Punk-Problem
fokusiert.\\

\subsubsection{Grundlegende Technik}
Der Algorithmus von Dijkstra löst das Eins-zu-Viele-Problem auf einem gewichteten gerichten Graphen.
Dabei ist zu beachten dass alle Kanten $e \in E$ nicht-negativ gewichtet sind, also $e(u,v) \geq 0$.

Der Algorithmus besteht aus einer Initialisierungsphase, in der die Kosten $dist$ für alle Knoten
auf den Wert $\infty$ gesetzt werden. Bereits besuchte Knoten werden in der Menge $M$ gespeichert,
um zu verhindern, dass Knoten doppelte besucht werden. Um am Ende nicht nur die Kosten, sondern auch
den Weg zu erhalten wird zusätzlich der Vorgänger $P$ für jeden Knoten gespeichert. In folgender
Implementierung wird eine Min-Vorrangwarteschlange $Q$ verwendet, in der zu Beginn der Startknoten $s$ mit
Kosten 0 eingefügt wird. In der Hauptschleife wird nun solange ein Knoten $u \in V - S$ aus der
Warteschlange entnommen, bis diese leer ist. Wenn ein Nachbarknoten $v$ von $u$ noch nicht besucht
wurde, werden die Kosten $dist(v)$ aktualisiert, falls der Weg über $u$ kürzer ist. Der Vorgänger
$P(v)$ wird ebenfalls aktualisiert und der Knoten $v$ in die Warteschlange eingefügt. Der
Algorithmus  terminiert, wenn alle Knoten besucht wurden. Die Laufzeit des Algorithmus beträgt
$\Theta(|V|^2)$, wenn eine Adjazenzmatrix verwendet wird, oder $\Theta(|E| + |V| \log |V|)$, wenn
Adjazenzlisten verwendet werden \cite{intro.algo}.\\

\begin{algorithm}[H]
	\caption{Algorithmus nach Dijkstra}
	\label{algo:dijkstra}
	\begin{algorithmic}
		\Function{Dijkstra}{G, s}
		\State Kosten $dist$ für alle Knoten außer $s$ mit $\infty$ bewerten
		\State dist(s) = 0
		\State Besuchte Knoten $M = \emptyset$
		\State Vorgänger $P = \emptyset$
		\State Q.push(s)

		\While{$Q \ne \emptyset$}
		\State u = Q.pop()
		\State $M = M \cup \{u\}$

		\For{alle ausgehenden Kanten $e(u,v) \in Adj[u]$ und v $\notin M$}
		\If{dist(u) + w(u,v) < dist(v)}
		\State dist(v) = dist(u) + w(u,v)
		\State P(v) = u
		\State Q.push(v)
		\EndIf
		\EndFor
		\EndWhile
		\EndFunction
	\end{algorithmic}
\end{algorithm}


\subsubsection{Zielorientiert}
\subsubsection{Hierarchisch}

\subsection{OpenStreetMap}